\documentclass {article}
\usepackage{amssymb}
\begin {document}
\title{\textbf{Laboratory Data Management System}}
\author{Harshit Maheshwari\\N V Subba Rao}
\date{August 9, 2013}
\maketitle
\newpage
%\tableofcontents
\newpage
%\section {Answer 1}
%\subsection{(a)}
\flushleft{
\LARGE{\textbf{Abstract}}\\[0.5cm]
\normalsize{

Scientists and engineers document research, experiments and procedures performed in laboratory using lab notebooks. Organizing and keeping a detailed account of this data is a big issue and challenge for the laboratories. Also, security and protecting the integirity of this data is a very important aspect. \\
So we propose to build a 'Laboratory Data Management System' to tackle these issues. While doing research and conducting experiments data can be safely stored in a structured and organized manner using the system.\\[1cm]
}}
\flushleft{
\LARGE{\textbf{Proposed Functionalities}}\\[0.5cm]
\normalsize{
\textbf{Supported Data Types}\\[0.1cm]
The system would allow the users to add data in the form of text, images and audio files. If time permits other files formats may also be supported.\\[0.3cm]
\textbf{Data Verification}\\[0.1cm]
Any data that is recorded by the laboratory researchers will be stored by the system in an encrypted format. After the priliminary data storage the lab data stored would be verified by the supervisor. If approved, the data will be added to a central repository with the virtual signature of the supervisor.\\[0.3cm]
\textbf{Revision Control}\\[0.1cm]
If any changes are to be made in the data that is added to the repository, then each revision is associated with a timestamp and the person making the change. To maintain the integrity of the data any changes, done by the user, will be reflected seperately in form of patches.\\[0.3cm]
\textbf{Security of Data}\\[0.1cm]
The data files will be stored in an encrypted format by the system to prevent unauthorized use or reproduction of copyrighted material. Encrypting the data files will help to protect them should physical security measures fail.


}}


\end{document}

